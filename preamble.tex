\setlength{\textwidth}{\fullwidth}
\setlength{\evensidemargin}{\oddsidemargin}
\addtolength{\textwidth}{-5truemm}
\addtolength{\oddsidemargin}{5truemm}

%package類
%%%%%%%%%%%%%%%%%%%%
\usepackage[utf8]{inputenc} % エンコーディングが UTF8 であることを明示する。
%文書内リンク
% OTF フォントを使えるようにし、複数のウェイトも使用可能にする。
% これがないと、Mac のヒラギノ環境で使われる角ゴが太すぎてみっともない。
%\usepackage[deluxe]{otf}

% OT1→T1に変更し、ウムラウトなどを PDF 出力で合成文字ではなくす
\usepackage[T1]{fontenc}

% uplatex の場合に必要な処理
\usepackage[prefernoncjk]{pxcjkcat} % アクセントつきラテン文字を欧文扱いにする

% Helvetica と Times を sf と rm のそれぞれで使う。
% default だとバランスが悪いので、日本語に合わせて文字の大きさを調整する。
\usepackage[scaled=1.05,helvratio=0.95]{newtxtext}

% 色
\usepackage[dvipdfmx]{color}
\usepackage[colorlinks=true,allcolors=blue]{hyperref}
\usepackage{pxjahyper}%文字化け防止

\usepackage{array,amsmath,amssymb,bm,cases}%数式でよく使うもの

\usepackage[dvipdfmx]{graphicx}%画像使用

\usepackage{url}

\usepackage{natbib}

\usepackage{ascmac}%itembox環境

\usepackage{enumerate}%enumerate環境

% bibliography を目次に追加
\usepackage[nottoc,notlot,notlof]{tocbibind}

%\usepackage{mathtools}%dcases環境

\usepackage[nooneline]{subfigure}
\subfiguretopcaptrue

%\usepackage{comment}

\allowdisplaybreaks[4] % 数式がページをまたがることを許す

\usepackage{listings,jvlisting} %日本語のコメントアウトをする場合jvlisting(もしくはjlisting)が必要
%ここからソースコードの表示に関する設定
\lstset{
  basicstyle={\ttfamily},
  identifierstyle={\small},
  commentstyle={\smallitshape},
  keywordstyle={\small\bfseries},
  ndkeywordstyle={\small},
  stringstyle={\small\ttfamily},
  frame={tb},
  breaklines=true,
  columns=[l]{fullflexible},
  numbers=left,
  xrightmargin=0zw,
  xleftmargin=3zw,
  numberstyle={\scriptsize},
  stepnumber=1,
  numbersep=1zw,
  lineskip=-0.5ex
}
%ここまでソースコードの表示に関する設定

% latexdiff
% 実際の修論には入れる必要なし
%DIF PREAMBLE EXTENSION ADDED BY LATEXDIFF
%DIF UNDERLINE PREAMBLE %DIF PREAMBLE
\RequirePackage[normalem]{ulem} %DIF PREAMBLE
\RequirePackage{color}\definecolor{RED}{rgb}{1,0,0}\definecolor{BLUE}{rgb}{0,0,1} %DIF PREAMBLE
\providecommand{\MyDIFadd}[1]{{\protect\color{blue}\uwave{#1}}} %DIF PREAMBLE
\providecommand{\MyDIFdel}[1]{{\protect\color{red}\sout{#1}}}                      %DIF PREAMBLE
%DIF SAFE PREAMBLE %DIF PREAMBLE
\providecommand{\MyDIFaddbegin}{} %DIF PREAMBLE
\providecommand{\MyDIFaddend}{} %DIF PREAMBLE
\providecommand{\MyDIFdelbegin}{} %DIF PREAMBLE
\providecommand{\MyDIFdelend}{} %DIF PREAMBLE
%DIF FLOATSAFE PREAMBLE %DIF PREAMBLE
\providecommand{\MyDIFaddFL}[1]{\MyDIFadd{#1}} %DIF PREAMBLE
\providecommand{\MyDIFdelFL}[1]{\MyDIFdel{#1}} %DIF PREAMBLE
\providecommand{\MyDIFaddbeginFL}{} %DIF PREAMBLE
\providecommand{\MyDIFaddendFL}{} %DIF PREAMBLE
\providecommand{\MyDIFdelbeginFL}{} %DIF PREAMBLE
\providecommand{\MyDIFdelendFL}{} %DIF PREAMBLE
%DIF END PREAMBLE EXTENSION ADDED BY LATEXDIFF


%%%%%%%%%%%%%%%%%%%%

%newcommand類
%%%%%%%%%%%%%%%%%%%%
\newcommand{\bs}{\symbol{92}} %backslash

\newcommand{\red}[1]{\textcolor{red}{#1}} %文字色赤

\newcommand{\blue}[1]{\textcolor{blue}{#1}} %文字色青

\newcommand{\green}[1]{\textcolor{green}{#1}} %文字色緑

\newcommand{\ured}[1]{\textcolor{red}{\underline{\textcolor{black}{#1}}}} %下線赤

\newcommand{\ugreen}[1]{\textcolor{green}{\underline{\textcolor{black}{#1}}}} %下線緑

\newcommand{\ublue}[1]{\textcolor{blue}{\underline{\textcolor{black}{#1}}}} %下線青
%%%%%%%%%%%%%%%%%%%%

\makeatletter

\def\seireki#1{\def\@seireki{#1}} % 西暦での年度
\def\supervisor#1{\def\@supervisor{#1}} % 担当教員
\def\SyozokuUniv#1{\def\@SyozokuUniv{#1}} %所属大学
\def\SyozokuGakubu#1{\def\@SyozokuGakubu{#1}} %所属学部
\def\SyozokuGakka#1{\def\@SyozokuGakka{#1}} %所属学科
\def\StudentIdNumber#1{\def\@StudentIdNumber{#1}} % 学籍番号
\def\subject#1{\def\@subject{#1}} %教科名

\renewcommand{\maketitle}{%
  \begin{titlepage}%
    \let\footnotesize\small
    \let\footnoterule\relax
    \let\footnote\thanks
    \null\vfil
    \vskip 20\p@
    \begin{center}%
      {\Large もし天{\@seireki}年度版 \par}%
      \vskip 12em%
      {\LARGE \@title \par}%
      \vskip 15em%
      {\large
        \begin{tabular}[t]{c}%
          \@SyozokuUniv 天文学教室
        \end{tabular}\par}%
      \vskip 5em%
      {\large
        \begin{tabular}[t]{c}%
            {\Large \@author}
        \end{tabular}\par}%
      \vskip 1.5em
      {\large \@date \par}%
    \end{center}%
    \par
    \@thanks\vfil\null
  \end{titlepage}%
  \setcounter{footnote}{0}%
  \global\let\thanks\relax
  \global\let\maketitle\relax
  \global\let\@thanks\@empty
  \global\let\@author\@empty
  \global\let\@date\@empty
  \global\let\@title\@empty
  \global\let\title\relax
  \global\let\author\relax
  \global\let\date\relax
  \global\let\and\relax
}%

\makeatother

%%ここからを自分で書く
%%%%%%%%
\title{天文データ解析概論}
\date{\today} %日付
\seireki{2021} %年度
\author{天文B3\,SLA一同} %名前

\SyozokuUniv{東北大学} %所属大学
%%%%%%%%
%%ここまでを自分で書く