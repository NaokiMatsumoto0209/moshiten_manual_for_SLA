\setlength{\textwidth}{\fullwidth}
\setlength{\evensidemargin}{\oddsidemargin}
\addtolength{\textwidth}{-5truemm}
\addtolength{\oddsidemargin}{5truemm}

%package類
%%%%%%%%%%%%%%%%%%%%
\usepackage[utf8]{inputenc} % エンコーディングが UTF8 であることを明示する。
%\usepackage[deluxe]{otf}
% OTF フォントを使えるようにし、複数のウェイトも使用可能にする。
% これがないと、Mac のヒラギノ環境で使われる角ゴが太すぎてみっともない。
\usepackage[T1]{fontenc}
% OT1→T1に変更し、ウムラウトなどを PDF 出力で合成文字ではなくす
\usepackage[prefernoncjk]{pxcjkcat} % アクセントつきラテン文字を欧文扱いにする
% uplatex の場合に必要な処理
\usepackage[scaled=1.05,helvratio=0.95]{newtxtext}
% Helvetica と Times を sf と rm のそれぞれで使う。
% default だとバランスが悪いので、日本語に合わせて文字の大きさを調整する。
\usepackage[dvipdfmx]{color} % 色
\usepackage[colorlinks=true,allcolors=blue]{hyperref}
%\usepackage{hyperref} % 紙に印刷するときは青文字リンクは消す。
\usepackage{pxjahyper} % 文字化け防止
\usepackage{array,amsmath,amssymb,bm,cases} % 数式でよく使うもの
\usepackage[dvipdfmx]{graphicx} % 画像使用
\usepackage{url}
%\usepackage{natbib}
\usepackage[numbers]{natbib} % bibliographystyleを「jplain」にした時、エラーが出るならこちらを試す
\usepackage{ascmac} % itembox環境
\usepackage{enumerate} % enumerate環境

% bibliography を目次に追加
\usepackage[nottoc,notlot,notlof]{tocbibind}

%\usepackage{mathtools}%dcases環境

\usepackage[nooneline]{subfigure}
\subfiguretopcaptrue

%\usepackage{comment}

\allowdisplaybreaks[4] % 数式がページをまたがることを許す

\usepackage{listings,jvlisting}
%ここからソースコードの表示に関する設定
\lstset{
  basicstyle={\ttfamily},
  identifierstyle={\small},
  commentstyle={\smallitshape},
  keywordstyle={\small\bfseries},
  ndkeywordstyle={\small},
  stringstyle={\small\ttfamily},
  frame={tb},
  breaklines=true,
  columns=[l]{fullflexible},
  numbers=left,
  xrightmargin=0zw,
  xleftmargin=3zw,
  numberstyle={\scriptsize},
  stepnumber=1,
  numbersep=1zw,
  lineskip=-0.5ex
}
%ここまでソースコードの表示に関する設定

% latexdiff
% 実際の修論には入れる必要なし
%DIF PREAMBLE EXTENSION ADDED BY LATEXDIFF
%DIF UNDERLINE PREAMBLE %DIF PREAMBLE
\RequirePackage[normalem]{ulem} %DIF PREAMBLE
\RequirePackage{color}\definecolor{RED}{rgb}{1,0,0}\definecolor{BLUE}{rgb}{0,0,1} %DIF PREAMBLE
\providecommand{\MyDIFadd}[1]{{\protect\color{blue}\uwave{#1}}} %DIF PREAMBLE
\providecommand{\MyDIFdel}[1]{{\protect\color{red}\sout{#1}}}                      %DIF PREAMBLE
%DIF SAFE PREAMBLE %DIF PREAMBLE
\providecommand{\MyDIFaddbegin}{} %DIF PREAMBLE
\providecommand{\MyDIFaddend}{} %DIF PREAMBLE
\providecommand{\MyDIFdelbegin}{} %DIF PREAMBLE
\providecommand{\MyDIFdelend}{} %DIF PREAMBLE
%DIF FLOATSAFE PREAMBLE %DIF PREAMBLE
\providecommand{\MyDIFaddFL}[1]{\MyDIFadd{#1}} %DIF PREAMBLE
\providecommand{\MyDIFdelFL}[1]{\MyDIFdel{#1}} %DIF PREAMBLE
\providecommand{\MyDIFaddbeginFL}{} %DIF PREAMBLE
\providecommand{\MyDIFaddendFL}{} %DIF PREAMBLE
\providecommand{\MyDIFdelbeginFL}{} %DIF PREAMBLE
\providecommand{\MyDIFdelendFL}{} %DIF PREAMBLE
%DIF END PREAMBLE EXTENSION ADDED BY LATEXDIFF


%%%%%%%%%%%%%%%%%%%%

%newcommand類
%%%%%%%%%%%%%%%%%%%%
\newcommand{\bs}{\symbol{92}} %backslash
\newcommand{\red}[1]{\textcolor{red}{#1}} %文字色赤
\newcommand{\blue}[1]{\textcolor{blue}{#1}} %文字色青
\newcommand{\green}[1]{\textcolor{green}{#1}} %文字色緑
\newcommand{\ured}[1]{\textcolor{red}{\underline{\textcolor{black}{#1}}}} %下線赤
\newcommand{\ublue}[1]{\textcolor{blue}{\underline{\textcolor{black}{#1}}}} %下線青
\newcommand{\ugreen}[1]{\textcolor{green}{\underline{\textcolor{black}{#1}}}} %下線緑
%%%%%%%%%%%%%%%%%%%%

%%ここからを自分で書く
%%%%%%%%
\title{もし天における天体観測と解析のすすめ}
\date{\today} %日付
\author{東北大天文学部生SLA一同} %名前
%%%%%%%%
%%ここまでを自分で書く