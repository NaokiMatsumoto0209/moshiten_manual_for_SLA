\chapter{天体観測序論}
\label{chap_2}
ここでは、天体観測の基本事項について説明します。特に天体からの放射については、「なぜ放射が起こるか」などといった問題には量子力学が関係していたり、その伝達には輻射輸送や相対論的電磁気学(ここではいわゆるラジプロ(Radiative Processes in Astrophysics)のこと)が関係していたりと、多くの物理の知識が必要です。

そのため、ここでは基礎的な説明にとどめますので、さらに深く勉強したい、という場合には、それぞれに適した文献を参照してください。

\section{天文における天体観測}
\label{sec_2_1}
天文学における観測の方法、またはその対象にはどのようなものがあるのでしょうか。天文学では、世間一般の方が思い浮かべるような物理の実験のように、天体のそばに行って情報を得たり\footnote{太陽系内については例外です。}、実際に触ったりすることができません。ニュースなどで取り上げられる全ての天文学の観測成果は、主に以下のような限られた対象の観測によってなされています。

\begin{enumerate}[1.]
    \item \textbf{電磁波}\\
    電磁波は大まかに言えば光のことであり、もっとも一般的な観測の手法となります。今回もし天で扱う観測対象です。詳しくは\ref{sec_2_2}~節で説明します。
    \item \textbf{重力波}\\
    ノーベル賞にもなった重力波は、最近観測できるようになったものです。中性子星などの超高質量天体の合体などで生じ光速で伝播する、アインシュタインによって予言された現象です。光速で伝播する、ということは、可視光観測などで突発的な変化が確認されたのちに速やかに観測を開始すれば、その現象を重力波の側面でも観測できるということになります。
    \item \textbf{宇宙線}\\
    \item \textbf{ニュートリノ}\\
    \item \textbf{サンプルリターン}\\
\end{enumerate}

\section{天体からの放射}
\label{sec_2_2}
そもそもなぜ天体は光を放つのか、どんな種類の光が存在するのでしょうか。

\subsection{電磁波}
\label{sec_2_2_1}
小学生の理科の授業などで、太陽光をプリズムで分光した経験がある方がいるかと思います。これは光が波長毎に異なる屈折率を持つことによって生じる現象です。高校の物理の最後の方でやった方もいるかと思いますが、光は
\begin{align}
    \mathrm{c} = \lambda \nu
\end{align}
と、光速$\mathrm{c}$と波長$\lambda$、周波数$\nu$で表されます。

\subsection{連続光とスペクトル線}
\label{sec_2_2_2}


\subsection{撮像観測と分光観測}
\label{subsec_2_2_3}

\section{データ処理とデータ解析}
\label{sec_2_3}
    