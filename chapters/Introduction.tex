\chapter{はじめに} %
\label{chap:introduction}

\section{作成経緯と目標} %%
\label{sect:text_goal}
\textbf{もし天}とは、東北大学天文学教室、宮城教育大学、加速キッチン合同会社、仙台市天文台が主催している高校生向けの研究体験事業で、全国の宇宙、科学に興味のある高校生たちが天文学者としての研究を体験するものです。2022年で12年目となり、多くのもし天卒業生、通称「もしチル」が多方面で活躍されています。また、SLA(Student Learning Assistant)の参加者の中にも、もしチルの方々が多く在籍しています。\par
そんな「もし天」ですが、もし天に限らず、天文学における観測は\textbf{単に写真を撮る}ことにとどまりません。観測の後には必ずその\textbf{データ}が存在し、データがどんな意味を持つのかを読み取るために、その解析を行う必要があります。\par
このテキストは、解析を行うSLAの方のために、光・赤外における観測とは何か、撮像観測と分光観測、また観測画像の処理等についてを解説し、その負担を軽減する目的で作られたものです。\par
解析の方法にはデータ毎に様々な方法があります。「もし天」で行う観測としては主に「\textbf{撮像観測}」と「\textbf{分光観測}」があります。これらの観測データの解析には歴史的に「\textbf{IRAF}」\cite{iraf}という、アメリカ国立光学天文台(NOAO)が開発した画像解析ソフトが使われてきました。もし天でもこれまで、このソフトを使って画像解析を行ってきました。\par
しかし先日、IRAFのNOAOの公式のサポートが今後行われない、つまりディスコンになるという発表がありました。今後の天文学の研究では、IRAFではなくPythonの「Astropy」モジュールを使用した解析が行われていくと考えられます。このAstropyは、現状撮像観測には対応していますが、分光観測には対応していません。そのため、分光観測にはIRAFを使い続けることになりますが、撮像観測にはもし天での解析についてもPythonへの移行が必要となります。\par
また、もし天に参加されたことのある方ならご存知かもしれませんが、もし天の実習期間中の観測可能な時間は非常に限られたものです。その中で、参加高校生の研究テーマに見合うだけの膨大な量のデータの解析を行わなければならず、毎年、SLAが毎夜のように徹夜で作業している光景が当たり前となっていました。これはもし天の存続性という観点や、SLAの健康という観点からも大変問題であり、解決する必要があります。\par
以上の理由から、
\begin{itemize}
    \item 「撮像観測でのPythonを使った解析」と「分光観測でのIRAFを使った解析」のわかりやすい解説
    \item SLAの負担軽減
\end{itemize}
を目指した解説資料が必要との判断から、東北大天文のSLAが解説資料を作成することとなりました。

\section{テキストの構成} %%
\label{sect:text_program}
このテキストでは主に以下のことを解説しています。
\begin{itemize}
  \item 天文学の基本 $\rightarrow$ \ref{chap:fandamentals_of_astronomy}~章
  \item 天文学における可視光観測の基本 $\rightarrow$ \ref{chap:fandamentals_of_observations}~章
  \begin{itemize}
    \item 撮像観測
    \item 分光観測
  \end{itemize}
  \item もし天における観測 $\rightarrow$ \ref{chap:moshiten}~章
  \item 天文学における解析の基本 $\rightarrow$ \ref{chap:fandamentals_of_analysis}~章
  \begin{itemize}
    \item 撮像データ解析
    \item 分光データ解析
  \end{itemize}
  \item 天文学におけるデータ考察の基本 $\rightarrow$ \ref{chap:fandamentals_of_discussion}~章
  \item 付録 $\rightarrow$ \ref{chap:appendix}~章
  \begin{itemize}
    \item プログラミング環境の構築
    \item Python実行環境の構築
    \item IRAF実行環境の構築
    \item バグ取りの翁
    \item Makali'i実習
    \item SAOImage DS9実習
  \end{itemize}
\end{itemize}

\section{掲載内容についての諸注意} %%
\label{sect:caution}
このテキストで解説されている内容は、東北大学天文学教室の学部3年向け通年授業「天体観測」の講義内容や、2011年度のもし天観測班の方の資料を参考に、書籍等を参照して肉付けしたものになっています。また、添付のPythonコードや、別資料の解析コードについては、「天体観測」の講義内容や、その担当教員である板先生の作成したものや、板研究室の資料を参考にしたものとなっています。

\textbf{従って、文書、コードについては板さんをはじめとした作成者に、著作権を要求しないような広く一般に普及した部分を除いて権利が帰属しますので、無断で外部への配布を行ったり、公的な文書への転載を行ったりしないように注意してください。}これは、万が一このテキストの内容に間違いがあったりした場合に責任を取りきれなくなってしまうためです。

また、この文書には文書内の相互リンク機能がついています。具体的には「\ref{chap:fandamentals_of_astronomy}~章」の数字部分をクリックすると、\ref{chap:fandamentals_of_astronomy}~章の初めのページにリンクされるようになっています。その他にも節のラベルや、図、数式のラベルにも同様のリンクがついています。適宜活用して頂けると幸いです。