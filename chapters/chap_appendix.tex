\chapter*{付録}
\label{chap_appendix}
\addcontentsline{toc}{chapter}{付録} % 目次に載せる

\setcounter{section}{0} % section の番号をゼロにリセットする
\renewcommand{\thesection}{\Alph{section}} % 数字ではなくアルファベットで数える
\setcounter{equation}{0} % 式番号を A.1 のようにする
\renewcommand{\theequation}{\Alph{section}.\arabic{equation}}
\setcounter{figure}{0} % 図番号
\renewcommand{\thefigure}{\Alph{section}.\arabic{figure}}
\setcounter{table}{0} % 表番号
\renewcommand{\thetable}{\Alph{section}.\arabic{table}}

\section{Pythonのモジュールのインストールに関連したsomething}
\label{sec_app_1}
\ref{sec_3_3}~節で、必要なモジュールのインストールについて説明しました。ここでは、この段階でエラーが出た場合の説明を行います。おそらくですが、Google Colaboratoryを使っている場合はそのようなことにはならないと思います。

もし天に限らず、あらゆるプログラミングについて、エラーメッセージを読むことは大切なことです。例えば、「numpyがない」というようなエラーが出た場合は、
\begin{verbatim}
python -m pip install numpy
\end{verbatim}
とすればnumpyがインストールされます。その他のモジュールについても同様で、特定のバージョンをインストールしたい場合は
\begin{verbatim}
python -m pip install nanntoka==1.0.0
\end{verbatim}
などとします。アップデートの要求をされた場合は
\begin{verbatim}
python -m pip install --upgrade nanntoka
\end{verbatim}
とします。ここら辺はネット検索するといくらでも解説記事が出ますので、\textbf{エラーメッセージをよく読んで}、それに合わせた検索をかけて見てください。\footnote{まあエラーを読んですぐに原因がわかるようなら苦労しません。最初にエラーを全部そのままコピペしてネットで検索するのが吉だと思います。}
