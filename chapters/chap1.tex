\chapter{はじめに} %
\label{chap_1}
\section{作成経緯}%
\label{sec_1_1}
このテキストは、高校生向け研究体験事業「もしも君が杜の都で天文学者になったら」(通称「もし天」)において解析を行うSLAの方のために、光・赤外における観測とは何か、撮像観測と分光観測、また観測画像の処理等についてを解説し、その負担を軽減する目的で作られたものです。

「もし天」は2021年度現在、東北大学天文学教室、宮城教育大学、加速キッチン合同会社、仙台市天文台が主催している高校生向けの研究体験事業で、全国の宇宙、科学に興味のある高校生たちが天文学者としての研究を体験するものです\cite{moshiten}。2021年で11年目となり、多くのもし天卒業生、通称「もしチル」が多方面で活躍されています。また、今年度のSLAにも過去にもし天に参加された、もしチルの方々が多く在籍されています。

そんな「もし天」ですが、もし天に限らず、天文学における観測は\textbf{単に写真を撮る}ことにとどまりません。観測の後には必ずその\textbf{データ}が存在し、データがどんな意味を持つのか、その解析を行う必要があります。

その解析の方法にはデータ毎に様々な方法があります。「もし天」で行う観測としては主に「\textbf{撮像観測}」と「\textbf{分光観測}」があります。これらの観測データの解析には歴史的に「\textbf{IRAF}」\cite{iraf}という、アメリカ国立光学天文台(NOAO)が開発した画像解析ソフトが使われてきました。もし天でもこれまで、このソフトを使って画像解析を行ってきました。

しかし先日、IRAFのNOAOの公式のサポートが今後行われない、つまりディスコンになるという発表がありました。今後の天文学の研究では、IRAFではなくPythonの「Astropy」モジュールを使用した解析が行われていくと考えられます。このAstropyは、現状撮像観測には対応していますが、分光観測には対応していません。そのため、分光観測にはIRAFを使い続けることになりますが、撮像観測にはもし天での解析についてもPythonへの移行が必要となります。

また、もし天に参加されたことのある方ならご存知かもしれませんが、もし天の実習期間中の観測可能な時間は非常に限られたものです。その中で、参加高校生の研究テーマに見合うだけの膨大な量のデータの解析を行わなければならず、毎年、SLAが毎夜のように徹夜で作業している光景が当たり前となっていました。これはもし天の存続性という観点や、SLAの健康という観点からも大変問題であり、解決する必要があります。

以上の理由から、
\begin{itemize}
    \item 「撮像観測でのPythonを使った解析」と「分光観測でのIRAFを使った解析」のわかりやすい解説。
    \item SLAの負担軽減。
\end{itemize}
を目指した解説資料が必要との判断から、今年度の東北大天文の、学部3年SLAが解説資料を作成することとなりました。

\section{本テキストの構成}%
\label{sec_1_3}
このテキストでは、主に以下のことを解説しています。
\begin{enumerate}[1.]
    \item 天体観測序論 $\rightarrow$ \ref{chap_2}~章
    \item 撮像データ解析 $\rightarrow$ \ref{chap_3}~章
    \item 分光データ解析 $\rightarrow$ \ref{chap_4}~章
    \item 画像解析ソフト「Makali'i」解説 $\rightarrow$ \ref{chap_5}~章
\end{enumerate}

\ref{chap_2}~章の「天体観測序論」では、天体の放射に関する簡単な説明と、天文学における観測とデータ解析について説明します。この章は本当に基礎的な内容にとどめますので、天文学の観測について一通り学んだ方については読み飛ばして頂いて構わないかと思います。\ref{chap_3}~章の「撮像データ解析」では、もし天で主に行うことになる撮像データ解析について、特に各々の処理に着目しながら説明します。具体的なコードについては別資料を主に参考としていただければと思いますが、この資料でもコードを補助的に記載しますので、適宜ご活用いただければと思います。\ref{chap_4}~章の「分光データ解析」では、IRAFを用いたスペクトル分光について解説します。分光については前述の通り、現状Pythonで行うことができません。また、IRAFはMacでしか使うことができません。\ref{chap_5}~章の「画像解析ソフト「Makali'i」解説」では、国立天文台が開発した画像解析ソフト「Makali'i\cite{makali}」の使用方法を解説します。

最後に、付録(\ref{chap_appendix})として予想されるエラーや、その他解説事項について補足します。

\section{本テキストの内容について}%
\label{sec_1_4}
このテキストで解説されている内容は、東北大学天文学教室の学部3年向け通年授業「天体観測」の講義内容や、2011年度のもし天観測班の方の資料を参考に、書籍等を参照して肉付けしたものになっています。また、添付のPythonコードや、別資料の解析コードについては、「天体観測」の講義内容や、その担当教員である板先生の作成したものや、板研究室の資料を参考にしたものとなっています。

\textbf{従って、文書、コードについては板さんをはじめとした作成者に、著作権を要求しないような広く一般に普及した部分を除いて権利が帰属しますので、無断で外部への配布を行ったり、公的な文書への転載を行ったりしないように注意してください。}これは、万が一このテキストの内容に間違いがあったりした場合に責任を取りきれなくなってしまうためです。

また、この文書には文書内の相互リンク機能がついています。具体的には「\ref{chap_1}~章」の数字部分をクリックすると、\ref{chap_1}~章の初めのページにリンクされるようになっています。その他にも節のラベルや、図、数式のラベルにも同様のリンクがついています。適宜活用して頂けると幸いです。

\section{Pythonの環境構築について}%
\label{sec_1_5}
この節では、学部1、2年生の方や他大の方で、Pythonをこれまで扱う経験がなかった方のために、Pythonの環境構築について軽く紹介します。ただし、環境を合わせるという観点や、仙台市天文台の観測データの都合上、玄人の方にも「Google Colaboratory」をお勧めします。

\subsection{Google Colaboratory}%
\label{subsec_1_5_1}
「Google Colaboratory」\footnote{\url{https://colab.research.google.com}}はGoogle社提供のPythonの無料対話型実行環境です。ここでGoogle Colaboratoryをお勧めする理由は、仙台市天文台からの観測データがGoogle Driveに入っているから、という単純な理由です。ローカルでの特別な設定は必要ないので、自分のGoogleアカウントでログインして使用できます。別でお渡しする解析資料でもGoogle Colaboratoryを使用しています。

\subsection{Anaconda}%
\label{subsec_1_5_2}
もし天に限らず、将来的にもPythonを使った科学技術計算をしたいという場合に便利なのが、Anaconda社提供の「Anaconda」です。Anacondaをインストールすると、科学技術計算で必要なパッケージをまとめてインストールしてくれるだけでなく、環境についてもチームメンバーで簡単に共有できます。

自分で辿るとなると面倒なので、URLを添付します。
\begin{verbatim}
    https://www.anaconda.com/products/individual
\end{verbatim}
\begin{figure}
    \centering
    \includegraphics[width=0.6\linewidth]{fig/chap1/anaconda.jpeg}
    \caption[Anacondaのインストール]{Anacondaのインストール画面。自分の環境にあったものを選ぶ。今回はWindowsとMacであれば「Graphical installer」、Linuxは筆者が詳しくないので自分で調べて欲しい。\label{fig_1_1}}
\end{figure}

このページの下の方に、図~\ref{fig_1_1}のようなリンクがあるので、自分の環境にあったものを選択します。選択するとすぐにダウンロードが開始されるので、ダウンロードが終了したら、そのファイルを実行します。あとはライセンスへの同意を求められたり、インストール先の確認が行われるので、そこら辺は適宜行ってください。
\begin{figure}
    \centering
    \includegraphics[width=0.6\linewidth]{fig/chap1/anaconda_2.jpeg}
    \caption[Anaconda Navigator]{Anaconda Navigator。仮想環境を作ってバージョン別の環境を作ったりと、様々なアレンジができる。環境の情報は他人と共有できるので、プロジェクトで同一の環境を使いたい場合などに便利。 \label{fig_1_2}}
\end{figure}
    
画像処理を行うのであれば、決まった処理を一度全てに適用して流せば良いので、「Spyder」が便利かと思います。しかし、処理ごとに確認しながら、対話型で作業したいのなら「jupyter notebook」、またはその後継の「jupyter lab」が便利かもしれません。その辺りは個人によると思いますので、色々試してみてください。

\section{IRAFの環境構築について}
\label{sec_1_6}
IRAFは前述の通り、そのままではMacでしか利用することができません。Windowsを使用している方などは、Macを余分に所有している方からお借りする必要があります。どうしても、という場合はお借りすることになりますが、そもそも解析は一人で行うものではないと思うので、Macを持っている人に分光解析は任せて、撮像解析を担当するなど、様々なやり方があると思います。

IRAF自体は
\begin{verbatim}
https://iraf-community.github.io
\end{verbatim}
から現在は有志によるサポートが行われているものをダウンロードできます。IRAFの使用のためにはxgtermの用意など、準備が必要です。

\subsection{xgtermについて}
\label{subsec_1_6_1}

\subsection{SAOimage DS9について}
\label{subsec_1_6_2}

\subsection{初回起動時のmkiraf}
\label{subsec_1_6_3}